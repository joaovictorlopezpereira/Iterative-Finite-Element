\maketitle
\newpage
\begin{center}
  \textbf{Resumo:}
\end{center}

  Este documento apresenta como trabalho final da disciplina \textit{Computação Científica e Análise de Dados} a resolução de um sistema genérico de equações diferenciais parciais com condições de Dirichlet utilizando o Método dos Elementos Finitos, conforme abordado nas aulas da disciplina \textit{Introdução ao Método dos Elementos Finitos}, ministradas na Universidade Federal do Rio de Janeiro, durante o segundo semestre de 2024.

  O método dos elementos finitos é amplamente utilizado para transformar equações diferenciais parciais e ordinárias em sistemas lineares da forma $Ax = b$, que podem ser resolvidos por métodos numéricos iterativos, como o método de Gauss-Jacobi. Esse método é particularmente interessante para esse contexto devido à estrutura da matriz $A$, que, nesse caso, é esparsa e tri-diagonal, o que reduz significativamente o custo computacional e o uso de memória, tornando-o adequado para sistemas de grande escala. No documento, também é realizada uma análise detalhada das vantagens do método de Gauss-Jacobi nesse cenário, considerando suas propriedades e desempenho em sistemas esparsos gerados pelo método dos elementos finitos.

\vspace{0.3cm}

\begin{center}
  \textbf{Abstract:}
\end{center}

  This document presents the solution of a generic system of ordinary differential equations with Dirichlet conditions using the Finite Element Method, as covered in the course \textit{Introduction to the Finite Element Method}, taught at the Federal University of Rio de Janeiro, during the second half of 2024, as the final project for the course \textit{Scientific Computing and Data Analysis}.

  The finite element method is widely used to transform partial differential equations into linear systems of the form $Ax = b$, which can be solved using iterative numerical methods such as the Gauss-Jacobi method. This method is particularly advantageous in this context due to the sparse and tri-diagonal structure of the matrix $A$, significantly reducing computational cost and memory usage, making it suitable for large-scale systems. The document also provides a detailed analysis of the advantages of the Gauss-Jacobi method in this scenario, focusing on its properties and performance in the sparse systems generated by the finite element method.


\newpage \tableofcontents
\newpage
